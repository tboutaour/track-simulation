% !TEX root=MemoriaTFG.tex

\chapter{La memòria del treball de final de grau}\label{memoria}

\section{Principis bàsics}

\emph{Què és el que fa que una memòria de \ac{TFG} sigui bona?} Òbviament, la resposta a aquesta pregunta pot ser molt complexa, però hi ha molts autors (vegeu, per exemple, \cite{Perelman01,Malvar08} i els treballs que aquests referencien) que coincideixen en que un informe tècnic o científic ha de reunir les qualitats següents:
\begin{itemize}
   \item \emph{precisió}: aquesta qualitat es fonamenta en tres aspectes complementaris:
   \begin{enumerate}

   \item \emph{precisió del document}, que fa referència a que aquest s'ha de concentrar de forma precisa en el tema que el defineix i n'ha de fer un tractament correcte, basat en evidències científicament demostrables, i amb un nivell de detall apropiat, ni massa general ni excessivament restringit.

   \item \emph{precisió estilística}, que només és possible si es fa un ús curós del llenguatge per tal d'expressar el significat. La precisió en l'ús del llenguatge es fonamenta en el rigor en la utilització de les paraules, en una estructura correcta de les frases i en un ús adequat dels signes de puntuació.

   \item \emph{precisió tècnica}, que es fonamenta en un bon coneixement tècnic de la matèria i del seu vocabulari.
   \end{enumerate}

   \item \emph{concisió}: molt relacionada amb la precisió del document, la concisió es fonamenta en la focalització, és a dir, en la reducció de l'abast del document a l'àmbit estricte de la qüestió que es vol tractar. És molt fàcil caure en la temptació d'incloure en la memòria materials que potser són molt rellevants en el camp en què es desenvolupa el treball, però que no ho són en absolut per comunicar de manera efectiva les aportacions concretes del treball en aquest camp.
       %Com veurem més endavant, la definició precisa i detallada de l'índex de la memòria és una de les estratègies més útils per controlar la longitud i l'abast del document. També és important identificar i eliminar els materials (frases, paràgrafs, seccions, \ldots) que no són necessaris per recolzar o evidenciar les nostres aportacions. Les figures i les taules solen contribuir a la concisió del document perquè redueixen la quantitat de prosa necessària per descriure objectes i processos, resumeixen dades i serveixen per demostrar relacions.
%
   \item \emph{claredat}: aquesta qualitat, que fa referència a la capacitat de qui escriu per facilitar la comprensió del document a qui llegeix, és especialment important en l'escriptura científico-tècnica. Els vocabularis especialitzats, els desenvolupaments matemàtics o els esquemes conceptuals complexos poden dificultar moltíssim la comprensió d'algunes explicacions tècniques, fins i tot quan aquestes han estat redactades per escriptors especialitzats i són processades per lectors experts.

   \item \emph{coherència (organització i estructura)}: un document és coherent si el material que presenta està organitzat de manera lògica i consistent i la seva estructura proporciona al lector un camí fàcil per a la seva comprensió. La coherència es valora de forma molt especial en ciència i tecnologia degut a la inherent complexitat dels temes que es tracten.


   \item \emph{adequació a l'audiència}: és convenient que el document s'adeqüi als objectius que l'autor s'ha marcat a l'hora d'escriure'l, però, sobretot, és molt important que s'adeqüi a les necessitats dels possibles lectors (supervisors, membres del tribunal, altres alumnes que treballin en temes semblants, \ldots) i al context, objectius i convencions (forma i estil) de la institució en què es presenta.
\end{itemize}

Tal com ens recorda H. S. Malvar a \cite{Malvar08}, l'error més freqüent que podem cometre a l'hora d'escriure, especialment en el cas de l'escriptura científica, és no saber posar-nos al lloc del lector. Així doncs, la recomanació fonamental per evitar errors a l'hora d'escriure la memòria és que cada cop que afegim una nova argumentació, una nova figura o una nova taula, pensem en els possibles lectors d'aquesta informació i la revisem per tal de garantir que és precisa, clara, concisa, coherent i adequada a l'audiència.

\section{Bones pràctiques}

\emph{Quines són les millors passes a fer per escriure una bona memòria de \ac{TFG}?} La major part dels treballs de final de grau es desenvolupen en tres etapes, una primera etapa en què es duu a terme una revisió de les referències bibliogràfiques més rellevants sobre el context general i l'àmbit particular del projecte, una segona etapa en què es realitzen estudis analítics, simulacions, implementacions reals, \ldots i, finalment, una tercera etapa en què es redacta la memòria. Tanmateix, aquestes tres etapes no són estrictament consecutives i, habitualment, treuen profit d'una retroalimentació sistemàtica entre elles. En aquesta secció, tot i ser molt conscients d'aquesta interdependència, ens centrarem en l'etapa de redacció de la memòria i presentarem una sèrie de recomanacions a seguir en els processos de \emph{focalització}, \emph{elaboració de l'esbós} i \emph{redacció}.

\subsection{Focalització} Aquest procés aborda de ple les qualitats de precisió del document, concisió i adequació a l'audiència. La focalització consisteix en la reducció de l'abast del document a l'àmbit estricte que el defineix, en la selecció acurada dels temes sobre els que ha de tractar i en l'adequació dels continguts i profunditat de tractament d'aquests a l'audiència concreta a la que es vulgui dirigir. Per focalitzar un document, algunes de les passes que ens podrien servir serien:
\begin{enumerate}
   \item Fer una llista completa dels temes o paraules clau que defineixen el nostre treball.

   \item Si encara no ho hem fet, recórrer a les referències bibliogràfiques que calgui per tal de tenir una visió prou acurada del paper que juga cadascun d'aquests temes en el context general i en l'àmbit particular del tema que ens ocupa.

   \item A partir de la llista de temes i de la informació disponible sobre cadascun d'aquests, respondre a preguntes del tipus:
   \begin{itemize}
          \item És necessari que tracti aquest tema en la memòria?
          \item Es perdran els lectors de la memòria si no tracto aquest aspecte?
          \item Puc deixar de parlar sobre aquest tema sense perjudicar l'objectiu global del meu projecte?
          \item És excessivament general aquest tema o és excessivament específic?
          \item N'hi ha prou amb un tractament general acompanyat de referències bibliogràfiques o seria millor que el lector disposés d'una visió més detallada per tal de poder seguir els raonaments posteriors?
          \item És necessari introduir coneixements previs per poder tractar alguns dels temes de la llista?
   \end{itemize}
   Aquest tipus de preguntes actuaran com a filtres que ens permetran eliminar temes innecessaris, afegir-ne de necessaris o preveure la profunditat amb la que s'ha de tractar cadascun dels temes seleccionats. En haver acabat disposarem d'una bona eina per passar a l'etapa d'elaboració del primer esbós de la memòria.
\end{enumerate}

\subsection{Elaboració de l'esbós} Un cop hem focalitzat l'àmbit estricte del treball i, per tant, disposem de la llista de temes seleccionats i tenim una idea general de la profunditat amb què els volem tractar, una bona manera de facilitar el procés de redacció consisteix en la preparació d'un esbós detallat del nostre treball. Es comença amb una taula de continguts que conté un llistat dels títols provisionals dels capítols, seccions i subseccions que es volen incloure en la primera versió de la memòria. Després, per cadascun dels capítols, seccions i subseccions es redacten una sèrie de frases curtes que descriuen, de forma ordenada, els aspectes clau de tots i cadascun dels continguts que s'hi tractaran i, si es creu convenient, les fonts en les que ens podem recolzar en el procés de redacció. En aquestes etapes continua sent especialment important tenir en ment les qualitats de precisió, concisió i adequació a l'audiència de la nostra proposta, però també hi jugarà un paper especialment rellevant la qualitat de coherència. Paga la pena dedicar temps i esforç en l'elaboració d'aquest primer esbós i revisar-lo tantes vegades com sigui necessari fins a estar-ne totalment convençuts.

Un cop tenim l'esbós provisional del document és el moment de fer-ne una revisió acurada amb l'ajut del nostre supervisor. Aquesta revisió ens hauria de permetre, entre d'altres coses, eliminar el material innecessari, afegir el material o argumentacions que se'ns hagin pogut oblidar i reestructurar els continguts per tal d'incrementar el nivell de coherència del document. És més eficient i menys dolorós prendre aquestes decisions en les fases inicials del procés de redacció, que haver-les de prendre un cop ja s'ha escrit molt de material que al final s'ha de descartar. A partir d'aquest primer esbós ens resultarà més senzill visualitzar l'estructura global de la memòria i saber, en cada moment, quin és l'estat del procés de redacció. Ens permetrà determinar si el treball realitzat (simulacions, implementacions, anàlisis, \ldots), els resultats obtinguts i les referències bibliogràfiques consultades són suficients o si és necessari aprofundir en algun aspecte concret. Ens ajudarà, també, a planificar la nostra feina en funció del temps disponible.

\subsection{Redacció} L'estratègia de redacció a seguir a partir d'aquest primer esbós no ha de ser necessàriament lineal, és a dir, no ha de començar necessàriament amb la redacció del capítol d'introducció i acabar amb la del capítol de conclusions. De fet, tot i que és molt important partir d'una idea molt clara dels continguts del capítol d'introducció, atès que aquest és el que proporciona una visió global del context i l'abast del treball, en la majoria dels casos ens pot resultar més senzill començar amb la redacció dels capítols corresponents al desenvolupament del treball i acabar amb la redacció de la introducció i les conclusions. És l'anomenada estratègia \emph{de dintre cap a fora}.

És molt possible que, a mesura que anem redactant aquesta primera versió de la memòria, també descobrim que és necessari fer addicions, supressions i altres esmenes sobre l'esbós original a fi d'assolir-ne la versió definitiva. En qualsevol cas, atès que les frases curtes que hem utilitzat apunten, de forma ordenada i coherent, els aspectes clau de tots i cadascun dels continguts dels capítols, seccions i subseccions de la memòria, si ens dediquem a desenvolupar-les en forma de paràgrafs, recorrent, sempre que sigui necessari, a l'ajut de figures, taules, desenvolupaments matemàtiques, \ldots, al final obtindrem la primera versió de les diferents parts de la memòria.

La primera versió d'una secció o d'un capítol és la primera passa cap a la versió definitiva, però no ens ha de fer mandra revisar-la i reescriure-la tants cops com sigui necessari per tal de millorar-ne les qualitats de precisió, claredat, concisió, coherència i adequació. Aquest procés de revisió s'ha de fer a consciència, intentant descobrir paraules, frases, taules, figures o, fins i tot, paràgrafs que no contribueixen a que el text sigui més precís o més concís o més clar, procurant, també, millorar l'organització i l'estructura dels diferents elements que conformen el text i, no menys important, mirant d'aconseguir la precisió estilística a través de la correcció ortogràfica i gramatical.

La paraula clau és revisió i, com no podia ser d'altra manera, un cop donem per acabada la primera versió d'un capítol és el moment de que el nostre supervisor també la revisi. Atès que ja havíem mantingut reunions amb el supervisor per discutir l'esbós de la memòria, és poc probable que aquesta revisió suposi canvis substancials en l'estructura general del capítol. Tanmateix, hi pot haver aspectes del treball que només s'hagin tractat de manera més o manco informal i, per tant, la lectura d'aquesta versió pot ser el primer contacte formal del supervisor amb alguns dels plantejaments teòrics o pràctics realitzats per l'alumne. Així doncs, hi ha la possibilitat de que es detectin mancances o errors que suposin el replantejament d'alguns apartats. És molt important que considerem tots els suggeriments que ens faci el supervisor i que els utilitzem per millorar la següent versió de la memòria. Si s'han produït replantejaments d'algunes parts del treball pot ser necessària una segona revisió per part del supervisor.

Quan el supervisor hagi revisat tots els capítols de la memòria i nosaltres hàgim realitzat tots els canvis oportuns, només restarà portar a terme una revisió global de la memòria per tal d'obtenir-ne la versió definitiva.

\section{Estructura del Treball Final de Grau}

Tal com ens recorda el professor Valiente \cite{Valiente96}, organismes internacionals d'estandardització com l'ANSI (\emph{American National Standards Institute}) o la ISO (\emph{International Organization for Standardization}) prescriuen sistemes estàndard per a l'organització dels treballs científics que contenen quatre parts fonamentals: introducció, desenvolupament, resultats i conclusions. Aquestes parts fonamentals d'un treball científic es complementen amb altres components com la portada, la taula de continguts, les llistes de figures, taules i acrònims, el resum, els agraïments, els apèndixs o les referències bibliogràfiques.


\subsection{Portada}

La portada actua com a element de presentació i identificació del \ac{TFG}. Les dades que s'hi han de fer constar poden variar en funció del tipus de treball, però n'hi ha algunes que són fonamentals: títol, autors, directors i, si s'escau, tutors, departament, universitat, títol acadèmic al qual s'opta i data de presentació. Després de la portada s'acostumen a deixar un o més fulls en blanc de cortesia.

El títol és sens dubte una part molt important de la memòria del \ac{TFG}. És el primer lligam que s'estableix entre el \ac{TFG} i el lector i, per tant, s'ha de ser molt curós a l'hora de seleccionar les paraules i les frases que donaran forma al títol. Un bon títol és aquell que descriu el contingut del \ac{TFG} de manera precisa i amb el menor nombre de paraules possible. Una bona estratègia a l'hora d'escriure el títol del \ac{TFG} és partir d'una llista de paraules clau i tractar de trobar quines d'aquestes són fonamentals a l'hora de descriure la nostra aportació i quin ha de ser l'ordre i l'associació entre aquestes paraules clau. Òbviament, a mesura que anem escrivint la memòria podem anar manejant una sèrie d'alternatives que puguin donar lloc a diferents títols i deixar que amb el temps alguna d'elles es vagi imposant sobre les altres.

\subsection{Taula de continguts o Sumari}

La taula de continguts o sumari és un llistat dels títols dels diferents capítols, seccions i subseccions del document amb indicació dels números de pàgina en què apareixen. Per tant, la taula de continguts no només ajuda als lectors a cercar els diferents temes tractats en la memòria, sinó que també serveix com a esbós de l'estructura de la memòria i ofereix una visió general del document als lectors potencials. Òbviament, les taules de continguts més útils es componen de títols de caire descriptiu.

\subsection{Llista de figures}

Els lectors utilitzen la llista de figures per localitzar la informació visual en la memòria. La llista de figures relaciona els títols o llegendes dels recursos visuals (figures, dibuixos, fotografies, \ldots) amb la seva ubicació dintre de la memòria. És important que les llegendes de les figures siguin descriptives i que estiguin numerades de manera consecutiva.

\subsection{Llista de taules}

La llista de taules proporciona les llegendes i la localització de totes les taules que apareixen a la memòria. De la mateixa manera que els títols de figura, els títols o llegendes de les taules s'han de numerar consecutivament en l'ordre en què apareixen al document.

\subsection{Llista d'acrònims}

Hi ha documents que utilitzen una quantitat important de termes nous, o molts acrònims i abreviacions. En aquests casos es pot facilitar la lectura de la memòria si s'inclou una llista de nomenclatura o una llista d'acrònims just després de les llistes de figures i/o taules. Un dels efectes secundaris interessants de les llistes de nomenclatura o d'acrònims és que ajuden a l'autor del document a utilitzar la terminologia d'una manera coherent.

\subsection{Resum}

El resum és una breu declaració, generalment entre 250 i 500 paraules, que proporciona al lector una sinopsi del problema, el mètode, els resultats i les conclusions de la memòria. Els resums s'han de poder llegir de manera totalment independent de les altres parts de la memòria i, per tant, no s'hi han d'utilitzar acrònims sense definir-los i tampoc s'hi han d'utilitzar referències bibliogràfiques. Els resums són extremadament útils per aquelles persones que volen tenir una imatge general del contingut de la memòria abans de llegir el document principal. Atès que hi pot haver lectors potencials que utilitzin el resum per decidir si han de continuar llegint la memòria o no, cal no menystenir la importància d'una bona redacció d'aquest apartat del document. Els resums poden estalviar una immensa quantitat de temps als possibles lectors.

El resum ha d'incloure, com a mínim, els següents elements:
\begin{itemize}\tightlist
   \item Definició abreujada del problema o tema principal del \ac{TFG}.

   \item Exposició del mètode utilitzat per resoldre el problema.

   \item Comentaris sobre els principals resultats, aportacions i possibles aplicacions del treball.

   \item Conclusions més importants del treball
\end{itemize}

\subsection{Agraïments}

A vegades s'inclou una secció d'agraïments en els preliminars de la memòria per tal de donar crèdit a l'assistència rebuda de part de persones i/o institucions. Els supervisors, els tècnics de laboratori o els companys de feina que ens han assessorat o ens han donat suport són, tots ells, candidats a aparèixer al capítol d'agraïments.

\subsection{Introducció}

Si hem decidit utilitzar l'estratègia \emph{de dintre cap a fora}, després de redactar els capítols corresponents al desenvolupament del treball estarem en disposició d'enllestir la redacció de la introducció i les conclusions.

La introducció hauria de servir per donar, de forma descriptiva i fàcil d'entendre, una visió global del context i l'abast del treball. De fet, segons Booth~\emph{et al.}~\cite{Booth08}, el patró comú que cerquen els lectors en qualsevol introducció està format per tres elements:
\begin{itemize}
   \item Contextualització: es tracta d'explicitar el context en el que s'emmarca el treball, d'establir la base comuna de coneixements sobre el tema que es tractarà en la memòria del \ac{TFG}, de garantir que els possibles lectors comparteixen amb l'autor de la memòria el conjunt de factors que els permetran interpretar adequadament els seus enunciats i raonaments.

       El context d'un \ac{TFG} podria ser, per exemple, el món de les xarxes de comunicacions mòbils de quarta generació (4G), amb capes físiques basades en l'ús de múltiples antenes en transmissió i en recepció (\acsu{MIMO} -- \emph{\acl{MIMO}}), tècniques de codificació i modulació adaptatives (\acsu{AMC} -- \emph{\acl{AMC}}) i estratègies d'accés múltiple basades en l'ús de transmissió multiportadora (\acsu{OFDMA} -- \emph{\acl{OFDMA}}). En aquest cas seria adequat parlar sobre l'estat actual del desenvolupament dels estàndards 4G, de les característiques generals de les tecnologies \ac{MIMO}, \ac{AMC} i \acsu{OFDM}/\ac{OFDMA} i de la seva adequació als sistemes 4G. Depenent de l'àmbit d'aplicació del problema a tractar podria ser adequat aprofundir, per exemple, en la descripció de l'estat actual de les xarxes de comunicacions ce\lgem{}lulars i de les característiques de les estacions base i/o de les estacions repetidores o en la descripció de l'estat actual de les xarxes d'àrea local sense fils i de les possibles estratègies de cooperació entre punts d'accés.

   \item Definició del problema: un cop establert el context del treball, és l'hora de definir amb precisió el problema que es tractarà en el \ac{TFG} i de justificar la seva importància dintre del context en el que s'emmarca. Es tracta, també, de proporcionar una visió general, però concisa, dels antecedents del problema, de les publicacions més rellevants sobre el tema, de les virtuts i mancances dels plantejaments i solucions aportades per altres autors.

       Dintre del context de l'exemple anterior, un possible problema a resoldre en un \ac{TFG} podria ser, atesa la necessitat de gestionar els recursos disponibles d'una manera eficient, el de l'assignació òptima de potència, subportadores i modes de transmissió (codificació de canal i modulació) per tal de garantir una taxa de transmissió global màxima amb restriccions sobre la qualitat de servei proporcionada a les aplicacions dels diferents usuaris del sistema.

   \item Resposta al problema: després de definir el problema, el més lògic és presentar al lector de la memòria la nostra proposta per solucionar-lo, els nostres objectius. Es tracta d'explicitar el mètode seleccionat per solucionar el problema plantejat anteriorment, tot justificant aquesta selecció. També pot ser adequat avançar, tot i que de manera concisa, els resultats principals del \ac{TFG} i les possibles conclusions que es desprenen dels resultats obtinguts.

       En el problema de l'exemple anterior, una possible resposta podria consistir en el desenvolupament d'algorismes d'optimització dual, en l'ús de la teoria de jocs o, per posar-ne un altre exemple, en l'ús d'aprenentatge estadístic (\emph{machine learning}). En cadascun d'aquests casos hauríem de justificar la tria feta i, si ens semblés adient, podríem parlar de quins són els resultats que mostrarem al lector en el desenvolupament de la memòria del \ac{TFG}.

\end{itemize}


Avui en dia, tant en la universitat com en l'empresa, és molt habitual que el \ac{TFG} formi part de projectes de recerca més amplis, de manera que pot ser difícil per al lector discernir quan és que l'autor descriu el seu treball personal i quan és que descriu una tasca realitzada per altres membres del grup de recerca. En aquests casos, és molt important que l'autor expliciti el millor possible quin ha estat exactament el seu paper dintre d'aquest projecte general.

Tot i que hi ha autors que no ho recomanen, una manera habitual d'acabar una introducció consisteix en presentar un resum de l'estructura de la memòria, avançant al lector quins són els continguts dels diferents capítols del \ac{TFG}.


\subsection{Desenvolupament}

Aquesta part de la memòria, que es pot estructurar en diversos capítols, tracta sobre la pròpia realització del treball i descriu el que s'ha fet, com s'ha fet, per què s'ha fet d'aquesta manera i no d'una altra, quins materials o eines s'han utilitzat o s'han hagut de desenvolupar, quina metodologia de treball i de validació s'ha seguit, \ldots

L'estructura, organització i contingut d'aquesta part de la memòria depenen en gran mesura del tipus de \ac{TFG}: empírics, estudis de casos, metodològics, teòrics, \ldots\ Tanmateix, el principi bàsic ha de ser proporcionar informació suficient perquè un lector ben informat pugui comprendre, reproduir i verificar els experiments o els desenvolupaments teòrics, tot evitant la simple repetició enciclopèdica de coneixements que es poden trobar a llibres, articles o altres documents de referència. Per exemple, quan la memòria del \ac{TFG} comenci amb els fonaments teòrics del problema plantejat a la introducció, el seu tractament, necessàriament sintètic, ha de mostrar l'elaboració personal de la informació manejada i s'ha de tenir molta cura de no caure en la simple còpia dels autors referenciats. A partir d'aquesta elaboració personal s'entendrà el fil argumental seguit per l'autor a l'hora d'arribar a la resolució del problema plantejat.

\subsection{Resultats i discussió}

Els resultats obtinguts en el \ac{TFG} constitueixen la nostra contribució al coneixement científic. Així, doncs, aquesta part de la memòria ha de descriure tota la informació generada en el desenvolupament del \ac{TFG}. No n'hi haurà prou amb presentar les dades juntament amb les estimacions sobre la seva precisió, també serà necessari interpretar-les i situar-les en context comparant-les amb les obtingudes per altres autors o utilitzant altres mètodes proposats a la literatura.

En els capítols dedicats a la presentació i discussió de resultats és habitual utilitzar figures i taules per tal de mostrar les dades d'una manera efectiva. És important parar molt d'esment en l'elaboració tant de les figures com de les taules i, també, que en el text fem referència explícita als resultats que hi presentem. Si el \ac{TFG} ha produït una gran quantitat de dades potser no cal presentar-les totes en els capítols de resultats i el més adequat és fer-ne una selecció acurada que ens permeti complir amb el propòsit d'extreure'n i fonamentar de forma rigorosa les conclusions del \ac{TFG} i poder-ne fer una presentació adequada als lectors. De fet, si es considera oportú, les dades que no apareguin en aquests capítols de resultats es poden fer avinents als possibles lectors en els apèndixs.

\subsection{Conclusions}

Tot i que el capítol de conclusions d'un \ac{TFG} pot començar amb un resum del context i de la definició del problema i passar després a analitzar la importància del treball realitzat i dels resultats obtinguts, les conclusions no s'han de limitar a tornar exposar el que ja s'ha presentat en els capítols anteriors i, a més, no s'ha de caure en la trampa de repetir el mateix que es va dir a la introducció \cite{Pierson97}. Per tant, el resum del context i de la definició del problema ha de ser molt breu i ens hem de concentrar en la interpretació dels resultats i en la identificació de les nostres contribucions. Les conclusions han de donar resposta a preguntes del tipus: Quines implicacions teòriques i/o pràctiques pot tenir el meu treball? Quin valor afegit suposa aquest treball dintre del corpus de coneixements de la meva disciplina? Què és el que coneixem ara que no se sabia al principi d'aquest treball? Quines mancances i quina utilitat tenen els resultats obtinguts? Què es pot fer a partir d'aquests resultats? Quines portes hem tancat i quines hem deixat obertes? Quines recomanacions podem fer als que vulguin continuar en aquesta línia?

Així, doncs, el capítol de conclusions hauria de:
\begin{itemize}\tightlist
   \item Recordar de manera concisa el context i la definició del problema i dels possibles objectius marcats en la introducció del \ac{TFG}.

   \item Argumentar les principals conclusions del \ac{TFG} i establir les possibles implicacions teòriques i les possibles aplicacions pràctiques dels resultats obtinguts.

   \item Remarcar què és el que ha quedat i el que no ha quedat demostrat en la memòria del \ac{TFG}.

   \item Incloure recomanacions específiques per a futurs treballs relacionats amb el problema tractat en aquesta memòria.
\end{itemize}


\subsection{Apèndixs}

Els apèndixs contenen aquella informació que, tot i ser interessant que estigui en la memòria del \ac{TFG}, per un motiu o altre no és apropiat que aparegui en el cos principal d'aquesta. Els motius poden ser molt diversos però gairebé tots ells estarien relacionats amb la continuïtat del fil conductor de l'argumentació presentada en el cos principal de la memòria. Per exemple, demostracions matemàtiques molt extenses, taules completes de les característiques dels models utilitzats en les simulacions, especificacions tècniques dels components, grans taules de dades o el codi font d'un algorisme, són bons candidats per posar en un apèndix.

\subsection{Referències bibliogràfiques}

El coneixement científic, com qualsevol altre, és acumulatiu i, per tant, és normal que a mesura que anem escrivint la memòria del \ac{TFG} ho fem recolzant-nos en llibres, articles de revista, articles publicats en les actes d'un congrés, treballs inèdits, \ldots d'altres autors. Això és completament ``legal'' (no podrem ser acusats de plagi) sempre que citem de forma adequada les fonts bibliogràfiques utilitzades. Aquestes referències bibliogràfiques serviran, entre d'altres coses, per:
\begin{itemize}\tightlist
   \item donar suport a les nostres reivindicacions o augmentar la credibilitat de les nostres argumentacions,
   \item referenciar els antecedents que ens han portat fins a la feina que presentem en aquest \ac{TFG},
   \item donar exemples de diferents punts de vista sobre un tema determinat,
   \item cridar l'atenció sobre una posició amb la que volem mostrar el nostre acord o desacord, o
   \item destacar una frase o un passatge especialment rellevant tot citant la font original.
\end{itemize}
