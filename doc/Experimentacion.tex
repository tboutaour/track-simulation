%!TeX root=MemoriaTFG.tex
\chapter{Experimentacion}
\section{Experimentación sin importación de datos reales}
En este apartatado se pretende realizar la muestra de resultados de la generación de 10 rutas creadas 
mediante la simulación sin datos amacenados. Al no tener datos almacenados dentro del sistema. La 
simulación corresponderá a una elección aleatoria del los diferentes caminos ya que, por defecto, todos 
los caminos tienen la misma probabilidad de tránsito a falta de introducir datos en el modelo.
Por otra parte la distancia punto a punto será una constante definida en 10 metros, por lo que no 
aparecerá variabilidad en la muestra.

Se han realizado una simulación de 10 rutas. El resultado son 10 ficheros \ac{GPX} que están situados 
en la carpeta X adjuntada con este documento.

{\color{red}{INSERTAR IMAGEN DE ANÁLISIS DE ESAS 10 RUTAS}}

\section{Experimentación con importación de datos reales}
En este apartado se pretende realizar la muestra de resultados de la generación de 10 rutas creadas 
mediante la simulación a partir de los datos almacenados. 
Para la simulación se han analizado 25 rutas de un mismo individuo sobre el territorio delimitado del 
Castell de Bellver, Mallorca, España.
Debido a la falta de muestra se ha tomado la decisión de realimentar al sistema con la misma muestra, de 
forma que el número de detecciones por segmento sea mayor. De esta forma se da lugar a una simulación 
con mayor probabilidad de replicar las decisiones tomadas por el individuo.

{\color{red}{INSERTAR IMAGEN DE MAPA DE CALOR DEL CASTILLO}}

{\color{red}{INSERTAR IMAGEN DE ANÁLISIS DE ESAS 10 RUTAS}}




